\documentclass[11pt, a4paper]{article}
\usepackage[left=2cm,text={17cm, 24cm},top=3cm]{geometry}

\usepackage{times}
\usepackage[czech]{babel}
\usepackage[utf8]{inputenc}

\def\UrlBreaks{\do\/\do/}
\usepackage{url}
\DeclareUrlCommand\url{\def\UrlLeft{<}\def\UrlRight{>} \urlstyle{tt}}
\def\UrlBreaks{\do\/\do-}
\usepackage[breaklinks,hidelinks]{hyperref}
\usepackage{breakurl}



\begin{document}

\begin{titlepage}
\thispagestyle{empty}

    \begin{center}

        {\Huge \textsc{Vysoké učení technické v~Brně \\[0.5em]}}

        {\huge \textsc{Fakulta informačních technologií}}

        \vspace{\stretch{0.382}}

        {\LARGE Typografie a publikování\,--\,4. projekt \\[0.4em] 
         Niečo málo o typografii}

        \vspace{\stretch{0.618}}

    \end{center}
{\Large 2023 \hfill Boris Hatala (xhatal02)}

\end{titlepage}

\section*{Úvod}
Typografia je prostriedok, ktorý dáva myšlienke vizuálnu formu.~\cite{ambrose2005basics} 
Má veľký vpliv na výsledný charakter dizajnu.
Môže mať neutrálnu povahu s cieľom informovať čitateľa alebo slúžiť ako nástroj, 
ktorým chceme vyvolať émocie.

\section*{Moderná doba}
Túto oblasť výrazne ovplyvnili osobné počítače.
Textové editory, ktoré poskytujú, markantne zjednodušili úpravu textu.~\cite{rybička1995latex}
To, na čo sme kedysi využívali zložitú technológiu a spotrebovali
sme pri tom množstvo materiálu, dnes zvládneme takmer s hocjakým počítačom.\\
Avšak benefity modernej doby prinášajú aj negatíva vo forme amatérizmu, teda vizuálneho
znečistenia, tak ako ho opísal pán Rajlich J.~\cite{ralich1998typografie}
\begin{quote} 
\uv{Vizuální znečištění pomíjí vžité estetické normy a pravidla, 
nezná základní principy tvorby a užití
prostředků vizuální komunikace, vypadá jednou jako kýč, podruhé jako škrabopis, ale někdy se
prohlašuje i za vrchol umění.}
\end{quote}
Vizuálne informácie tvorené diletantmi nespĺňajú 
účel a prispievajú ku \uv{\,grafickému smogu\,},
ktorý je veľmi rozpytľujúci.

\section*{Spôsob ako zaujať}
Prehľadnosť a lákavý dizajn je pri typografii naozaj kľúčová disciplína. 
Didaktická typografia je oblasť typografie, ktorá sa zaoberá sadzbou učebníc.
Jej cieľom je upútať pozornosť žiaka. Učebnica
by mala byť na pohľad zaujímavá, fascinujúca.~\cite{trousil2015didaktik}\par
S čím súvisí aj veľkosť písma. Poukazujeme ňou na dôležitosť informácií, 
ale ovplyvnňujeme ňou aj rýchlosť čítania.
Čím sú písmená menšie, tým rýchlejšie 
vedia naše oči nasledovať text.~\cite{legge2011size}\par 

Zaujať nechcú iba autori učebníc, ale aj firmy.
Výrazné logo tvorí komplexnú identitu značky.
Samotný dizajn môže napovedať o zameraní firmy, definovať jej charakter.~\cite{fuse}
Je to forma reklamy, pretože dobré logo je ľahko zapamätateľné.\par

Špecifické fonty alebo farby môžu	 propagovať aj ideológiu alebo dobu. 
Napríklad neónové pútače za času komunizmu boli často vo forme tzv. skriptov. 
Vzbudzovali v ľuďoch pocit dynamiky a napredovania kvôli ich šikmej povahe.~\cite{neon}
Červenej farbe sa na druhú stranu prikladal ideologicko-politický význmam, 
kvôli Sovietskému zväzu, ktorý ju využíval za účelom propagandy.~\cite{ideo}

\section*{Prístup k tvorbe}
Dizajnéry typov písma sú umelci ako maliari alebo muzikanti,
teda aj oni majú individuálny prístup k tvorbe. 
Niektorí modernizujú staré typy písma, dávajú im nový život.
Tento fenomén zažili napríklad serif fonty v roku 2020.~\cite{serif}\par
Ale dá sa to aj inak, tvorbou neštandardných typov písma, 
ktoré nevyhovujú bežným konvenciám. 
Touto cesto sa vybrala dizajnérka Natasha Lucas.
Jej abstraktné tvary porušujú mnohé pravidlá typografie
a možno práve preto sa im dostalo toľko obdivu.~\cite{abstract} 
Je v poriadku ignorovať pravidlá, pokiaľ sme sa ich už naučili dodržiavať.

\section*{Záver}
Najskôr sme sa pozreli na to ako typografia vplýva na naše okolie 
a ako nás môžu rôzne typy písma ovplivňovať. 
Nakoniec sme videli rôzne príklady fontov a aký kreatívny postup predchádzal jej tvorbe.


\newpage
\bibliographystyle{czechiso}
\bibliography{lib}

\end{document}