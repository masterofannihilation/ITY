\documentclass[11pt, a4paper, twocolumn]{article}
\usepackage[left=1.4cm,text={18.2cm, 25.2cm},top=2.3cm]{geometry}

\usepackage{times}
\usepackage[czech]{babel}
\usepackage[IL2]{fontenc}
\usepackage[utf8]{inputenc}

\usepackage{amsmath, amsthm, amssymb}
\usepackage[bottom]{footmisc}
\usepackage{graphicx}
\usepackage{hyperref}

\newtheorem{definition}{Definice}
\newtheorem{theorem}{Věta}

\begin{document}

\begin{titlepage}
\thispagestyle{empty}

    \begin{center}

        {\Huge \textsc{Vysoké učení technické v~Brně \\[0.5em]}}

        {\huge \textsc{Fakulta informačních technologií}}

        \vspace{\stretch{0.382}}

        {\LARGE Typografie a publikování\,--\,2. projekt \\[0.4em] 
        Sazba dokumentů a matematických výrazů }

        \vspace{\stretch{0.618}}

    \end{center}
{\Large 2023 \hfill Boris Hatala (xhatal02)}

\end{titlepage}

\newpage
\setcounter{page}{1}

\section*{Úvod}\label{Page:1}
V této úloze si vyzkoušíme sazbu titulní strany, matematických vzorců, 
prostředí a dalších textových struktur obvyklých pro technicky zaměřené texty -- například
Definice~\ref{def1} nebo rovnice~\eqref{Eq:3} na straně~\pageref{Page:1}. Pro vytvoření těchto odkazů používáme 
kombinace příkazů \verb|\label|, \verb|\ref|, \verb|\eqref| a \verb|\pageref|. 
Před odkazy patří nezlomitelná mezera. Pro zvýrazňování textu jsou zde několikrát
použity příkazy \verb|\verb| a \verb|\emph|.\par
Na titulní straně je použito prostředí \texttt{titlepage} a sázení nadpisu podle 
optického středu s~využitím \textit{přesného} zlatého řezu. 
Tento postup byl probírán na přednášce. Dále jsou na titulní straně použity
čtyři různé velikosti písma a mezi dvojicemi řádků textu je použito 
odřádkování se zadanou relativní velikostí 0,5\,em a 
0,4\,em\footnote[1]{Nezapomeňte použít správný typ mezery mezi číslem a jednotkou.}. 

\section{Matematický text}

V~této sekci se podíváme na sázení matematických symbolů a výrazů v~plynulém textu 
pomocí prostředí \texttt{math}. Definice a věty sázíme pomocí příkazu \verb|\newtheorem|
s~využitím balíku \texttt{amsthm}. Někdy je vhodné použít konstrukci \verb|${}$|
nebo \verb|\mbox{}|, která říká, že (matematický) text nemá být zalomen.

\begin{definition}\label{def1}
\textup{Zásobníkový automat} (ZA) je definován jako sedmice tvaru 
$A = (Q, \Sigma, \Gamma, \delta, q_0, Z_0, F ) $, kde:

\begin{itemize}
    \item
        $Q$ je konečná množina \textup{vnitřních (řídicích) stavů,} 
    \item
        $\Sigma$ je konečná \textup{vstupní abeceda,}
    \item
        $\Gamma$ je konečná \textup{zásobníková abeceda,}
    \item
        $\delta$ je \textup{přechodová funkce}
        $Q \times (\Sigma \cup \{\epsilon\}) \times \Gamma \rightarrow 2^{Q\times\Gamma^\ast}$,
    \item
        $q_0 \in Q$ je \textup{počáteční stav}, $Z_0 \in \Gamma$ 
        \textup{je startovací symbol zásobníku} a $F \subseteq Q$ je množina \textup{koncových stavů.}

\end{itemize}
\end{definition}

Nechť $P = (Q, \Sigma, \Gamma, \delta, q_0, Z_0, F )$ je ZA.
\textit{Konfigurací} nazveme trojici $(q, w, \alpha) \in Q \times \Sigma^\ast \times \Gamma^\ast$, kde $q$ je aktuální stav vnitřního řízení, $w$ je dosud nezpracovaná část vstupního
řetězce a $\alpha = Z_{i_1}Z_{i_2} \dots Z_{i_k}$ je obsah zásobníku.

\subsection{Podsekce obsahující definici a větu}

\begin{definition}\label{def2}
\textup{Řetězec} $w$ \textup{nad abecedou} $\Sigma$ \textup{je přijat ZA} A~jestliže 
$(q_0, w, Z_0) \underset{A}{\overset{*}{\vdash}} (q_F, \epsilon, \gamma)$
pro nějaké $\gamma \in \Gamma^\ast$ a $q_F \in F$. 
Množina $L(A) =$ \{$w\; |\; w$ je přijat ZA A\} $\subseteq \Sigma^\ast $ je
\textup{jazyk přijímaný ZA} A.
\end{definition}

\begin{theorem}
    Třída jazyků, které jsou přijímány ZA, odpovídá \emph{bezkontextovým jazykům.}
\end{theorem}

\section{Rovnice}
Složitější matematické formulace sázíme mimo plynulý text pomocí prostředí 
\texttt{displaymath}. Lze umístit i několik výrazů na jeden řádek, ale pak je třeba tyto vhodně
oddělit, například příkazem \verb|\quad|.
\begin{align*}
1^{2^3} \neq \Delta_{\Delta_{\Delta^3}^2}^1 \quad y_{22}^{11} - \sqrt[9]{x + \sqrt[7]{y} } \quad x > y_1 \leq y^2 
\end{align*}
V~rovnici \eqref{Eq:2} jsou využity tři typy
závorek s~různou \textit{explicitně} definovanou velikostí. Také nepřehlédněte,
že nasledující tři rovnice mají zarovnaná rovnítka, a použijte k~tomuto účelu vhodné 
prostředí.
\begin{align}
    -\cos^2 \beta \ \ &= \ \  \frac{\frac{\frac{1}{x} + \frac{1}{3}}{y} + 1000}{\underset{j=2}{\overset{8}{\prod}}\,q_j} \label{Eq:1}\\
    \biggl( \Bigl\{ b \star \bigl[3 \div 4\bigr] \circ a\Bigr\}^{\frac{2}{3}} \biggr) \ \  &= \ \ \log_{10} x \label{Eq:2}\\
    \int_a^b f(x)\,\mathrm{d}x \ \  &= \ \  \int_c^d f(y)\,\mathrm{d}y \label{Eq:3}
\end{align}
\noindent V~této větě vidíme, jak vypadá implicitní vysázení limity
$\lim_{m\to\infty} f(m)$
v~normálním odstavci textu. Podobn  ě je to i s~dalšími symboly jako
$\bigcup_{N\in\mathcal{M}} N$
či 
$\sum_{i=1}^{m} x_{i}^{2}$.
S~vynucením méně úsporné sazby příkazem \verb|\limits| 
budou vzorce vysázeny v~podobě $\lim\limits_{m \to \infty} f(m)$ a 
$\underset{i=1}{\overset{m}{\sum}} x_i^4$.

\section{Matice}
Pro sázení matic se velmi často používá prostředí \texttt{array}
a závorky (\verb|\left|, \verb|\right|).
$$
\mathbf{B} =\left| \begin{array}{cccc}
        b_{11} & b_{12} & \hdots & b_{1n} \\
        b_{21} & b_{22} & \hdots   & b_{2n} \\
        \vdots & \vdots & \ddots & \vdots  \\
        b_{m1} & b_{m2} & \hdots   & b_{2n} 
    \end{array} 
    \right|
     = \left| \begin{array}{cc}
        t & u \\
        v & w 
    \end{array}\right| = tw - uv 
$$
$$  
\mathbb{X} =\mathbf{Y} \Longleftrightarrow 
    \left[ 
    \begin{array}{ccc}
        & \Omega + \Delta & \hat{\psi} \\
        \vec{\pi} & \omega &
    \end{array} \right] 
    \neq 42
$$
Prostředí \texttt{array} lze úspěšně využít i jinde, například na pravé straně 
následující rovnice. Kombinační číslo na levé straně vysázejte pomocí příkazu
\verb|\binom|
\begin{align*}
    \binom{n}{k} = 
    \left\{ 
    \begin{array}{cl}
    0 & \textrm{pro}\, k < 0 \\
    \frac{n!}{k!(n-k)!} & \textrm{pro}\, 0 \le k \le n \\
    0 & \textrm{pro}\, k > 0
    \end{array} \right.
\end{align*}

\end{document}